\documentclass[12pt]{article}

\usepackage[utf8]{inputenc}
\usepackage[catalan]{babel}

\usepackage{amsmath}
\usepackage{xfrac}

\newcommand{\T}{\sfrac{1}{3}} % 1/3
\newcommand{\F}{\sfrac{1}{4}} % 1/4
\newcommand{\NH}{\sfrac{n}{2}} % n/2

\title{Exercici 26}

\begin{document}

\section{Resolució del problema}

\renewcommand{\thesubsection}{\alph{subsection}}

\subsection{Acotació de les normes de les matrius d'iteració}

La norma de $B_J$ es pot calcular explícitament:

\begin{equation*}
D^{-1}=
\begin{pmatrix}
    3       & 0       & \dots   & 0       \\
    0       & 4       & \ddots  & 0       \\
    \vdots  & \ddots  & \ddots  & \vdots  \\
    0       & 0       & \dots   & 4
\end{pmatrix}^{-1}=
\begin{pmatrix}
    \T      & 0       & \dots   & 0       \\
    0       & \F      & \ddots  & 0       \\
    \vdots  & \ddots  & \ddots  & \vdots  \\
    0       & 0       & \dots   & \F
\end{pmatrix}
\end{equation*}

\begin{equation*}
L+U=
\begin{pmatrix}
    0       & 0       & -1      & 0       & \dots   & 0       & 0       & 1       & 0       \\
    0       & 0       & 0       & -1      & \ddots  & 0       & 0       & 0       & 1       \\
    -1      & 0       & 0       & 0       & \ddots  & 0       & 0       & 0       & 0       \\
    0       & -1      & 0       & 0       & \ddots  & 0       & 0       & 0       & 0       \\
    \vdots  & \ddots  & \ddots  & \ddots  & \ddots  & \ddots  & \ddots  & \ddots  & \vdots  \\
    0       & 0       & 0       & 0       & \ddots  & 0       & 0       & -1      & 0       \\
    0       & 0       & 0       & 0       & \ddots  & 0       & 0       & 0       & -1      \\
    1       & 0       & 0       & 0       & \ddots  & -1      & 0       & 0       & 0       \\
    0       & 1       & 0       & 0       & \dots   & 0       & -1      & 0       & 0
\end{pmatrix}
\end{equation*}

\begin{align*}
\begin{gathered}
-D^{-1}(L+U)=\\\\
=-\begin{pmatrix}
    \T      & 0       & \dots   & 0       \\
    0       & \F      & \ddots  & 0       \\
    \vdots  & \ddots  & \ddots  & \vdots  \\
    0       & 0       & \dots   & \F
\end{pmatrix}
\begin{pmatrix}
    0       & 0       & -1      & 0       & \dots   & 0       & 0       & 1       & 0       \\
    0       & 0       & 0       & -1      & \ddots  & 0       & 0       & 0       & 1       \\
    -1      & 0       & 0       & 0       & \ddots  & 0       & 0       & 0       & 0       \\
    0       & -1      & 0       & 0       & \ddots  & 0       & 0       & 0       & 0       \\
    \vdots  & \ddots  & \ddots  & \ddots  & \ddots  & \ddots  & \ddots  & \ddots  & \vdots  \\
    0       & 0       & 0       & 0       & \ddots  & 0       & 0       & -1      & 0       \\
    0       & 0       & 0       & 0       & \ddots  & 0       & 0       & 0       & -1      \\
    1       & 0       & 0       & 0       & \ddots  & -1      & 0       & 0       & 0       \\
    0       & 1       & 0       & 0       & \dots   & 0       & -1      & 0       & 0
\end{pmatrix}=\\\\
=\begin{pmatrix}
    0       & 0       & \T      & 0       & \dots   & 0       & 0       & -\T     & 0       \\
    0       & 0       & 0       & \F      & \ddots  & 0       & 0       & 0       & -\F     \\
    \T      & 0       & 0       & 0       & \ddots  & 0       & 0       & 0       & 0       \\
    0       & \F      & 0       & 0       & \ddots  & 0       & 0       & 0       & 0       \\
    \vdots  & \ddots  & \ddots  & \ddots  & \ddots  & \ddots  & \ddots  & \ddots  & \vdots  \\
    0       & 0       & 0       & 0       & \ddots  & 0       & 0       & \T      & 0       \\
    0       & 0       & 0       & 0       & \ddots  & 0       & 0       & 0       & \F      \\
    -\T     & 0       & 0       & 0       & \ddots  & \T      & 0       & 0       & 0       \\
    0       & -\F     & 0       & 0       & \dots   & 0       & \F      & 0       & 0
\end{pmatrix}
\end{gathered}
\end{align*}

Per tant,
$\left\lVert D^{-1}(L+U)\right\rVert_\infty
=\max{\{\sfrac{2}{3},\sfrac{2}{4},\sfrac{2}{3},\sfrac{2}{4},\dots\}}
=\sfrac{2}{3}
<1$.
\\\\
Similarment, podem calcular $B_{GS}$ i acotar-ne la norma:

\begin{equation*}
(D+L)^{-1}=
\begin{pmatrix}
    \T      & 0       & 0       & 0       & \dots   & 0       & 0       & 0       & 0       \\
    0       & \F      & 0       & 0       & \ddots  & 0       & 0       & 0       & 0       \\
    \T^2    & 0       & \T      & 0       & \ddots  & 0       & 0       & 0       & 0       \\
    0       & \F^2    & 0       & \F      & \ddots  & 0       & 0       & 0       & 0       \\
    \vdots  & \ddots  & \ddots  & \ddots  & \ddots  & \ddots  & \ddots  & \ddots  & \vdots  \\
    0       & 0       & 0       & 0       & \ddots  & \T      & 0       & 0       & 0       \\
    0       & 0       & 0       & 0       & \ddots  & 0       & \F      & 0       & 0       \\
    \T^{\NH}& 0       & 0       & 0       & \ddots  & \T^2    & 0       & \T      & 0       \\
    0       & \F^{\NH}& 0       & 0       & \dots   & 0       & \F^2    & 0       & \F
\end{pmatrix}
\end{equation*}

Per altra banda, la matriu $U$ té norma infinit $2$.

Per tant,

\begin{equation*}
\left\lVert (D+L)^{-1}U\right\rVert_\infty
\leq\left\lVert (D+L)^{-1}\right\rVert_\infty \left\lVert U\right\rVert_\infty
=2\sum_{i=1}^{\NH}{\T^i}
<2\sum_{i=1}^{\infty}{\T^i}
=2\sfrac{1}{2}
=1
\end{equation*}

és a dir, el mètode convergeix.

\subsection{Resolució del sistema}

Les solucions del sistema donades pels tres mètodes estan emmagatzemades als fitxers
\texttt{jacobi.txt}, \texttt{gs.txt} i \texttt{sor.txt}, on cada fila es correspon
amb una entrada del vector solució.

El valor òptim de la constant $\omega$ que s'ha trobat és $1.2$.
Amb aquesta $\omega$, SOR resol el sistema en 34 iteracions, en comptes de les 41
que necessita Gauss-Seidel.

\section{Estructura del programa}
El programa està separat en tres fitxers:

\begin{description}
\item [\texttt{jacobi.c}] conté el codi per resoldre el sistema mitjançant el
  mètode de Jacobi.
\item [\texttt{gs.c}] conté el codi per resoldre el sistema mitjançant el
  mètode de Gauss-Seidel.
\item [\texttt{sor.c}] conté el codi per resoldre el sistema mitjançant el
  mètode de SOR. A més a més, conté la funció que s'ha utilitzat per trobar
  el valor òptim de la constant de SOR $\omega$.
\end{description}

Cada programa retorna per la sortida estàndard totes les components del vector
$x$, en ordre, una per línia.

S'inclou un \texttt{Makefile}. Dins del mateix s'explica cada instrucció.
En particular, la instrucció per defecte (\texttt{make}) genera els tres
programes, els executa, i emmagatzema les sortides en fitxers de text per
poder-les comparar.

\end{document}
