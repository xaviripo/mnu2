\documentclass[12pt]{article}

\usepackage[utf8]{inputenc}
\usepackage[catalan]{babel}

\usepackage{amsmath}

\title{Exercici 26}

\begin{document}

\section{Resolució del problema}

\renewcommand{\thesubsection}{\alph{subsection}}

\subsection{Acotació de les normes de les matrius d'iteració}

La norma de $B_J$ es pot calcular explícitament:

\begin{equation*}
\left(\begin{matrix}
    x_{11} & x_{12} & x_{13} & \dots  & x_{1n} \\
    x_{21} & x_{22} & x_{23} & \dots  & x_{2n} \\
    \vdots & \vdots & \vdots & \ddots & \vdots \\
    x_{d1} & x_{d2} & x_{d3} & \dots  & x_{dn}
\end{matrix}\right)
\end{equation*}

\begin{equation*}
D^{-1}=
\left(\begin{matrix}
    3       & 0       & \dots   & 0       \\
    0       & 4       & \dots   & 0       \\
    \vdots  & \vdots  & \ddots  & \vdots  \\
    0       & 0       & \dots   & 4
\end{matrix}\right)^{-1}=
\left(\begin{matrix}
    1/3     & 0       & \dots   & 0       \\
    0       & 1/4     & \dots   & 0       \\
    \vdots  & \vdots  & \ddots  & \vdots  \\
    0       & 0       & \dots   & 1/4
\end{matrix}\right)
\end{equation*}

\section{Estructura del programa}
El programa està separat en tres fitxers:

\begin{description}
\item [\texttt{jacobi.c}] conté el codi per resoldre el sistema mitjançant el
  mètode de Jacobi.
\item [\texttt{gs.c}] conté el codi per resoldre el sistema mitjançant el
  mètode de Gauss-Seidel.
\item [\texttt{sor.c}] conté el codi per resoldre el sistema mitjançant el
  mètode de SOR. A més a més, conté la funció que s'ha utilitzat per trobar
  el valor òptim de la constant de SOR $\omega$.
\end{description}

Cada programa retorna per la sortida estàndard totes les components del vector
$x$, en ordre, una per línia.

S'inclou un \texttt{Makefile}. Dins del mateix s'explica cada instrucció.
En particular, la instrucció per defecte (\texttt{make}) genera els tres
programes, els executa, i emmagatzema les sortides en fitxers de text per
poder-les comparar.

\end{document}
